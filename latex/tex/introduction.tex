\chapter{Introduction}

%\section{Problem Statement}
%\label{problem_statement}

Current debugging tools for robotics are mostly based on techniques and interfaces developed for traditional non-robotic applications. Those techniques and interfaces were developed specifically to suit the requirements of traditional systems: First, they assume that a program can be interrupted during its execution, which in a robotic environment is not possible most of the time. Second, the data handled in traditional applications is usually discrete and based on user input, as opposed to the sensory data a robot handles. The data a robot handles is in general closely related to the real world environment and comes from sensors that provide a continuous data stream of readings. Although the computing world has changed in recent years, many tools still rely on the original principles developed for different requirements. Additional tools have been developed to complement or extend traditional tools with robotic specific functionality \cite{Gumbley2010}. %drop last two sentences or rephrase?

% The application field of robotics has special requirements for debugging tools which are often not met by current debugging tools.

Traditional debugging tools usually render data collected during debugging as text and it is the developer's task to interpret the data in order to get a better understanding of the program and find possible bugs. A traditional software environment is usually suspendable and the execution of a program can be interrupted to give the developer enough time to interpret and analyse the debugging data.
On the other hand, it can be a problem to interrupt the execution of a program in a robotic environment, because robots generally do not run in such a deterministic and suspendable environment \cite{Gumbley2009}. Interrupting a robotic application would destroy the continuity in which the sensors collect data, the environment of the robot would change substantially and thus the robot's behaviour would change. This is an example of the ``probe effect'' and it makes it hard to reproduce a fault unless it has a single cause \cite{Gumbley2009}. It is necessary to collect data during debugging without interrupting the execution of the program, which means the developer has significantly less time to interpret and analyse the debugging data.

Murphy et al. found that despite the existence of tools to support debugging, many novice developers still use ``printf-style'' debugging \cite{Murphy2008}. They only examined novice debuggers but older studies indicate that both expert and novice debuggers use the same strategies for debugging \cite{Gugerty1986}. Debugging with print or logging statements is a basic approach, which does not require external tools, but to use it the source code must be modified. Source code modification can be a problem with so-called ``Heisenbugs'', software faults that disappear because the observation affected the bug \cite{Grottke2005}. The data collected with print or logging statements is usually text-only. This requires the developer to constantly parse and interpret logging messages. This results in a significant cognitive load for developers during debugging \cite{Jacobs2003}.
% Due to the large amount of data that is processed this often means developer consoles are filled with logging messages that often contain complex content such as numeric data. This requires a high degree of experience to focus on the data stream and the right moment in order to reveal a bug. %rephrase last sentence?

%This chapter introduces the problem statement for this work and gives an overview over the goals for this project. The last section of this chapter explains the structure of the chapters in this thesis.

%\section{Goals}
Since the robotics field has different requirements for debugging tools than the traditional software development field, many new tools have been developed and existing tools have been adapted to suit the special requirements. % fix inconsistency with first paragraph (check)
Despite the effort to create specialized tools for robotics, many developers still develop their own tool because existing tools are not flexible enough for their application area \cite{Collett2010}. %printf part is duplicated
To fulfil the special requirements for robotics and to counter the ``one-tool-per-project'' trend this work proposes a new kind of debugging system. The proposed system visualizes data during the debugging process in robotics and is flexible enough to be used for many different application areas. Existing debugging tools and methods were to be analysed first, to identify possible strengths and weaknesses of those tools and methods.

%this work is based on the assumption that...; add something about cognitive load; the scope of this work...evaluate the hypothesis
This work is based on the assumption that a flexible visual debugging system can support the developer during the debugging of a robotic application and reduce the cognitive load during debugging. It thus improves the development speed and developer satisfaction. Such a hypothesis can only be evaluated with an extensive user study, which is out of scope for this work. The purpose of this work is to design and develop such a flexible visual debugging system, which can be evaluated in the future. The development of such a system is a necessary step for further evaluations.

% ---> this is more or less exactly the same text as in the goals section for the visual debugging system
%This work presents a generic visual debugging system to tackle the robotics specific problems mentioned in the previous section. The system design is not based on a specific framework for robotics and the system can be implemented for any of the available frameworks. The visual debugging system aims to support the developer during debugging by visualizing data in a graphic way and thus eliminating the cognitive effort needed to parse and interpret text based logging messages. While most of the currently available visualization tools in robotics focus on spatial data to help understand the robot and the environment in which it runs \cite{Collett2010, Quigley2009}, rendering of abstract data is still uncommon. The system proposes a dashboard interface to robot developers, which they can populate with graphical widgets to visualize all kinds of data from the robot. The dashboard can be customized to display widgets according to the current robot hardware and development stage. It can be used to visualize data during debugging as well as monitor data during the normal execution of the robot. This means that the designed debugging system can be adapted to many different use cases and allows the developer to choose the widgets he or she thinks represent the data best, according to their mental model and the meaning of the data.

%ROSDashboard, the tool presented in this work, aims to support the developer during debugging by visualizing data in a graphic way and thus eliminating the cognitive effort needed to parse and interpret text based logging messages. While most of the currently available visualization tools in robotics focus on spatial data to help understand the robot and the environment in which it runs \cite{Collett2010, Quigley2009}, rendering of abstract data is still uncommon. ROSDashboard provides a dashboard interface to robot developers, which they can populate with graphical widgets to visualize all kinds of data from the robot. The dashboard can be customized to display widgets according to the current robot hardware and development stage. It can be used to visualize data during debugging as well as monitor data during the normal execution of the robot. This means ROSDashboard is a tool that a) can be adapted to many different use cases and b) allows the developer to choose the widgets he or she thinks represent the data best, according to their mental model and the meaning of the data. The tool is based on ROS (Robot Operating System) which abstracts from specific robot hardware and takes care of inter process communication \cite{Quigley2009}.

%\section{Outline}
The remainder of this work is structured as follows. Chapter~\ref{debugging_in_robotics} presents related tools and methods that are used to debug robotic applications. At the end of Chapter~\ref{debugging_in_robotics} the presented tools and methods are summarized and existing problems are outlined. Chapter~\ref{visual_debugging_system} presents the design of a generic visual debugging system for robotics. It is generic because it does not target a specific framework for robotics and can be implemented for any of the existing frameworks. The chapter contains goals and requirements for the debugging system which have driven the design and development. It also presents the architecture and system design of the proposed debugging tool in more detail. The presented design has been implemented in the ROSDashboard prototype which is described in detail in Chapter~\ref{rosdashboard}. ROSDashboard is a prototypical implementation of the proposed design for ROS (Robot Operating System) which represents one of the existing robotics frameworks. Chapter~\ref{case_study_chapter} contains a case study that shows the possibilities of the developed tool presented in Chapter~\ref{rosdashboard}. It also shows how the existing tool can be extended with more visualizations. The last chapter of this work, Chapter~\ref{future_work}, summarizes the possibilities to extend and improve ROSDashboard in the future and a summary of this project concludes the thesis.
