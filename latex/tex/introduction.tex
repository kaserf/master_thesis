\chapter{Introduction}
\review

Current debugging tools for robotics are mostly based on debugging techniques and interfaces developed for traditional non-robotic applications. Those techniques and interfaces were developed specifically to suit the requirements of traditional systems: First they assume that a program can be interrupted in its execution, which in a robotic environment is not possible most of the time. Second the data handled in traditional applications is usually discrete and based on user input, as opposed to the sensory data a robot handles. The data a robot handles is in general closely related to the real world environment of the robot and comes from sensors that provide a continuous data stream of readings. Although the computing world has changed in recent years, most tools have not. Robot developers often still use tools which were developed under different circumstances and based on different requirements. The application field of robotics has special requirements for debugging tools which are often not met by current debugging tools.

Traditional debugging tools usually render data collected during debugging as text and it's the developer's task to interpret the data. In a suspendable environment the developer has as much time as they need to interpret and analyse the data. When debugging robotic applications the developer usually has a lot less time, because of the nature of the application: Robotic applications usually can not be interrupted in their execution, because robots generally don't run in a deterministic and suspendable environment \cite{Gumbley2009}. Interrupting a robotic application would destroy the continuity in which the sensors collect data, the environment of the robot would change substantially and thus the robot's behaviour would change, which is an example of the ``probe effect'' and makes it hard to reproduce a fault unless it has a single cause \cite{Gumbley2009}. It is necessary to collect data during debugging without interrupting the execution of the program. Although some technical solutions have been presented in recent years \cite{Gumbley2009}, many developers still rely on printf-style debugging or other logging mechanisms. This approach is much simpler and does not require external tools, but the source code must be modified. Source code modification can be a problem with so called "Heisenbugs", software faults that disappear because the observation affected the bug \cite{Grottke2005}. The data collected with print or logging statements is usually text-only, which requires the developer to constantly parse and interpret logging messages. Due to the large amount of data that is processed this often means developer consoles are filled with logging messages that often contain more complex content such as numeric data.

ROSDashboard, the tool presented in this work, aims to support the developer during debugging by visualizing data in a graphic way and thus eliminating the cognitive effort needed to parse and interpret text based logging messages. While most of the currently available visualization tools in robotics focus on spatial data to help understand the robot and the environment in which it runs \cite{Collett2010, Quigley2009}, rendering of abstract data is still uncommon. ROSDashboard provides a dashboard interface to robot developers, which they can populate with graphical widgets to visualize all kinds of data from the robot. The dashboard can be customized to display widgets according to the current robot hardware and development stage. It can be used to visualize data during debugging as well as monitor data during the normal execution of the robot. This means ROSDashboard is a tool that a) can be adapted to many different use cases and b) allows the developer to choose the widgets he or she thinks represent the data best, according to their mental model and the meaning of the data. The tool is based on ROS (Robot Operating System) which abstracts from specific robot hardware and takes care of inter process communication \cite{Quigley2009}.

\section{Problem Statement}
\label{problem_statement}
[\textbf{problems with robotic software, many different environments, not interruptable, ...}]
Debugging robotic systems can prove to be much more complicated then normal software systems. 

\section{Goals (?)}
Maybe it would be good to explain the goals of this work in the introduction: 

\section{Outline}
Explain all the chapters and sections of this work.
