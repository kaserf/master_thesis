\chapter{Scenarios}
\dictum[John Dewey -- American philosopher, psychologist and educational reformer]
{\flushright{}Education is not preparation for life; education is life itself.}

\section{Problem Scenario}
In this section I would like to describe a scenario as it could happen in one
of the schools around the world. This scenario is the AS IS or problem
scenario, because it shows the problems I want to solve with the proposed
system which will be presented in chapter \ref{proposed_system_chapter}.

\begin{quote}
Lisa is a physics teacher at high school. She wants to use Phun \footnote{Phun
is a physics simulation game which is available for free at
http://www.phunland.com} with
her class to teach physics, especially Newton's laws of motion. After
a short introduction in Phun her pupils get a simple task: build an
inclined plane to accelerate a ball and let the ball crash into a
stack of boxes. The pupils should study the behavior of the boxes
and the ball and write their observations down. Some of her students
have been very fast in building the inclined plane, especially those
with some experience in similar applications (e.g. drawing applications).
But not all of her students are fluent in creating the right scenery
for the experiment. Bob has a problem in finding the right tools to
draw the inclined plane and gets more and more frustrated. At this
point his teacher Lisa joins him at his desk and helps him to resolve
his problem. In the meantime Martin has another Problem with his setup:
the inclined plane is too small and the ball cannot gain enough speed
to bring down the boxes. After Lisa has solved Bob's problem she goes over
to Martin in order to help him with his problem. Just after she has finished explaining
him the first steps to change the size of the plane the bell rings
and class is over.
\end{quote}

The problem is obvious: there is not enough time to support every
student with their individual problems. Martin cannot finish his work
in time because he gets stuck at some point and his teacher Lisa is
occupied helping someone else. The problem could be solved in a traditional
way by expanding the lessons and giving Lisa more time to support the children,
but that would only annoy the fast pupils. Another traditional way to solve this problem is to make smaller
classes or assign more teachers to one class. This would be really nice but is
not applicable in our schools because there are not enough teachers to make
smaller classes. A quite new and innovative solution for the problem would be
if the game (Phun in this case) itself could detect
Bob's and Martin's problems and support them in a way that they can
solve their problems themselves. Lisa could focus on more important
problems and Bob and Martin could finish their setup in time to make
some evaluations. 

%~backlog:
%~\begin{itemize}
%~\item as is scenario
%~\item describe the special problem related to the scenario, more detailed
%~then problem statement which is general
%~\item are there special actors: learner, peers, teacher, parents (?)
%~\end{itemize}

\section{Visionary Scenario}
\label{visionary_scenario}
The following scenario tells the same story as the Problem Scenario, but with the
improvements I would like to achieve using the proposed system. The scenario
describes an intelligent system built into the game that can help a child
with their problems. In terms of software engineering we can say that we have
one more actor in this scenario: the pedagogical agent. The pedagogical agent
is a non-player character (NPC) in the game which uses an intelligent
framework to understand the behavior of the learner (read more about pedagogical
agents in section \ref{pedagogical_agents}).

\begin{quote}
Lisa is a physics teacher at high school. [...]
Bob has a problem in finding the right tools to draw the inclined
plane and gets more and more frustrated. At this point the pedagogical
agent, Alice, who is Bob's agent of choice, pops
up after recognizing that Bob did not make any progress for some time.
Alice ``knows'' about the task as well and thus she highlights the
``plane tool'' to draw infinite planes and
displays the hint for that tool: ``When creating the
plane, hold down SHIFT to rotate it in 15 degree intervals, or just keep
the mouse close to the origin of rotation.'' Bob did
not notice the plane tool and is now aware of it and can use it.
At the same time somewhere else in the classroom Martin gets confronted
with his problem about the inclined plane being too small. Robert,
Martin's pedagogical agent, pops up because he noticed that Martin
tried the same setup over and over again and did not seem to be satisfied
with the outcome. Robert is aware of the problem situation because
he ``knows'' that the ball should bring down the boxes. Thus Robert brings
up a lecture about Newton's Laws of motion with the suggestion to either make
the ball bigger and heavier to have more mass or to make the inclined
plane bigger. Because for both solutions the ``resize
tool'' is needed, Robert highlights the tool in the toolbar.
\end{quote}

During the same amount of time, in which Lisa was able to help only one student (or even less),
the individual pedagogical agents helped both Martin and Bob with
their problems, and they are able to evaluate their setup in time to
write a short essay before the lesson is over. Of course the support of the pedagogical
agent will never be as good as the support of a human tutor, but it can still improve
the learning experience when used as an addition to a human teacher. In order to provide the
described support, the agent needs to understand the learner and know the goal of the
lesson and how it can be achieved.

%~The agents provided
%~soft scaffolding for both the students with different problems. Soft
%~scaffolding is only possible if the agents follow every action a student
%~performs and knows about the current lesson: what should be learned
%~and how it can be learned.

%~backlog:
%~\begin{itemize}
%~\item describe the hypothesis and the goal of the system, what makes the
%~scenario visionary
%~\item how to improve the problem scenario
%~\item what are the pedagogical and psychological principles behind the vision
%~\item what are the computer science principles
%~\item back up the vision with these principles
%~\end{itemize}
