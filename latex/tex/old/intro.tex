\chapter{Introduction}

\dictum[James Gee -- Member of the National Academy of Education]
{\flushright{}The biggest thing limiting games in education in my view is
the lack of good artificial intelligence to generate good and believable
conversations and interactions\dots{} We need games with expert systems built
into characters and the interactions players can engage in with the environment.
We need our best artificial tutoring systems built inside games, as well\dots{}
Then we will get games where the line between education and entertainment
is truly erased.}
%~\renewcommand*{\chapterpagestyle}{scrheadingstitle}
%~\pagestyle{scrheadings}

Political leaders have tried over and over again to discredit video and
computer games, although there is no scientific evidence supporting their
point of view. As a matter of fact most of the scientific findings lead
to a different outcome: video games are important for learning and can
be used at different academic levels \cite{DeAguilera2003}.
One of the advantages of games is the intrinsic motivation many teachers
would love to see with their students. Combining the motivation to
play games with teaching makes a powerful combination: intrinsically motivated
students.

The current digital gaming business is a \${10} billion per year industry
\cite{VanEck2006} with high expectations \cite{Susi2007}. Ben Sawyer, co-funder
of the ``Serious Games Initiative'', stated that the serious games market is
now at about \${20} million \cite{Susi2007}. This gives the whole gaming
in education concept a completely new perspective: business. In 2003 the global
education and training market was estimated at \${2} trillion \cite{Susi2007}.

To increase the learning effect of educational games, a recent study
tried to adapt the game to the gamer and in this case the learner
\cite{Kickmeier-Rust2006}. The researched aspects most often only cover
classic computer games, where the learner has to fulfill missions or
tasks in order to proceed and to finish the game. The focus of this work is on completely
different games. They can be called sandbox games or non-linear games,
which means that a player is completely free to explore and do what
they want to without being bond to a strict storyline. In these
kind of games it is difficult to monitor the learning effect and the
progress a learner makes. Examples for such games are physics simulations
like ``Phun''\footnote{http://www.phunland.com} or
``Physics''\footnote{http://wiki.laptop.org/go/Physics} and open-world health
games like ``Fatworld''\footnote{http://www.fatworld.org}. Open-world games for education are
inspired by some of the most successful computer games on the market like
the ``Grand Theft Auto'' (GTA) or ``Assasins Creed'' series.


Skinner already noted in his work ``The shame of American education''
\cite{Skinner1984a} that the
best learning outcome for a student can only be if the student
can proceed at individual speed. The optimal way to do this in school
classes would be if every student had their own teacher or at least
a tutor to help them \cite{Skinner1984a}. To make this possible, learning
games can be used to support the teacher: fast learners do not need
much help and can proceed alone, slower learners can have more
attention of the teacher because the teacher has more time for them.
In order to provide a good learning experience, the learning game should be
backed up by an intelligent framework. The intelligent framework I propose
later in this work is a framework which tries to understand the learners
behavior and support the learner accordingly. The crucial part is the problem
detection, once a problem is identified it can easily be overcome.
%~With an intelligent framework behind the game the goal is to understand
%~what the learner is up to and what its dificulties are. Once a problem
%~is identified it can easily be overcome.


\section{Serious Games}

Serious games (SG) have a higher purpose than pure entertainment: education. The term
refers to all kind of games: not only digital or computer games, but also
classic board games or outdoor games with educational purposes (see figure
\ref{taxonomy_games}). The term
``Serious Games'' first appeared as the title of Clark Abt's book,
which was originally published in 1970 by Viking Press \cite{Abt1987}. At the
time, serious games were mostly used by the military and health care sector to
provide training for soldiers and medical personnel.

\begin{figure}
    \centering
    \includegraphics[width=\textwidth]{diagrams/games_taxonomy.pdf}
    \caption[Taxonomy of Games (UML class diagram)]
    {Taxonomy of Games (UML class diagram)}
    \label{taxonomy_games}
\end{figure}

Although the term ``serious game'' was in use long before video and computer games
became popular, since 2002 it has been used in connection with electronic
games. The release of ``Americas Army'', a game produced by the US
Army for recruiting purposes, changed the ordinary use of the term. From that
moment on, ``serious games'' was used more often, mostly when a new game with a
learning background was to be promoted. In the same year of the release of
``Americas Army'' the ``Serious Games Initiative'' was
founded at the Woodrow Wilson International Center for Scholars.

In 2005 Michael Zyda proposed a definition of modern serious games: ``a mental
contest, played with a computer in accordance with specific rules, that uses
entertainment to further government or corporate training, education, health,
public policy, and strategic communication objectives.'' \cite{Zyda2005} This
definition applies to a vast field of concepts, which all have the same, or at
least similar, meanings: edutainment, e-Learning, game based learning and digital
game based learning \cite{Susi2007}. The difference between these concepts is
a topic of its own. Especially the difference between serious games (SG) and
(digital) games based learning (DGBL) is not clear, thus I choose to follow
Kevin Corti's example and use only the term serious game \cite{Corti2006}.

%~The difference between serious games (SG) and (digital) game based learning
%~(DGBL) is a topic with a lot of space to argue. Thus I choose to follow Kevin
%~Corti's example and use only the term serious games instead of making a
%~difference between SG and DGBL \cite{Corti2006}.

The concept of SG tries to keep the balance between learning outcome and gaming
experience. To distinguish itself from traditional e-Learning software, SG uses
state of the art computer game technology in combination with educational
sound concepts from the area of pedagogics and psychology to ensure the best
learning outcome. Both parts are equally important: a game that is fun but
is not able to educate the player is of no use for SG and in return a ``game''
that is not fun cannot be called a serious game either. Motivation is a
key factor in education. Serious games try to use the general motivation for
computer games to keep students playing and thus learning. Computer games as a
platform are used not only to harness the high motivational factors, they also
provide a powerful platform for children to develop a sense of autonomy and an
awareness of consequentiality \cite{Barab2005}. They also represent a good
opportunity to enlarge the so-called ``exploration space'' for
children, which has been continuously reduced over the last decades
\cite{Barab2005}.

%~Computer games in general are very good motivators. The proof for that is
%~nearly every teenager: useful to motivate children and learners in general, but not only that:
%~computer games also provide a powerful platform for children to develop a
%~sense of autonomy and an awareness of consequentiality \cite{Barab2005}. They
%~also represent a good opportunity to enlarge the so called ``exploration
%~space'' for children which has been continuously reduced over the last
%~decades \cite{Barab2005}.

%~\subsection{The Fun Factor}


%~SG don't focus on education only but also on fun. Fun is a key factor to keep the motivation
%~of the players - in this case learners or students - high, thus GBL
%~uses a powerful way to keep the learners playing the game and actually learning
%~things in the meantime.

%~Computer games are useful to motivate children and learners in general, but not only that:
%~computer games also provide a powerful platform for children to develop a
%~sense of autonomy and an awareness of consequentiality \cite{Barab2005}. They
%~also represent a good opportunity to enlarge the so called ``exploration
%~space'' for children which has been continuously reduced over the last
%~decades \cite{Barab2005}.


\section{Non-linear Games}

Non-linear is an adjective used for games with non-linear game play. Games
with linear game-play have a very clear story, which is better described as an interactive
video with missions which are aligned strictly. The player has few choices
to alter the flow and these choices only affect the success or failure of a
mission, but the order in which the missions can be completed is mostly
untouchable or can be adapted only slightly. Games that are non-linear give the
users exactly such choices, which allow them to alter the overall sequence of
events.

The term non-linear is ambiguous and due to the lack of a clear definition of
(serious) games, nearly every researcher uses other words which have a similar
or the same meaning as non-linear. Other terms which can be found in papers
and publications are \textbf{Open-World Games}, \textbf{Open-Ended
Games} or \textbf{Sandbox Games} and with a more educational background:
\textbf{Exploratory Learning Environment}. Drawing a line between these terms
makes little sense and hereafter the term non-linear games will be used,
which can be seen as a generalization of all the terms.

Non-linear games bring one big advantage over linear games: the player is even
more immersed in the game play, because they have the full freedom to explore
and control the ``virtual world'' they play in. This gives the player the feeling of ``being
there'' \cite{Psotka1995a}. Immersion is important in two different
aspects: first the educational aspect of immersion and then the motivational
factor that comes with immersive computer games. In recent studies, a high
correlation between immersion and addiction or high engagement in
computer games has been detected \cite{Seah2008a}.

With educational focus, there is another big advantage of non-linear games:
the adaptability of the game. Adaptation is important in order to focus on the
learner and give every student a personalized learning experience. The
importance of adaptation will be explained in more detail in Chapter 2, see
\ref{adaptation}. A non-linear game gives the player a lot more freedom and
thus the overall self-efficacy is higher, which can be very positive on
motivation and the learning outcome \cite{Zimmerman2000a}.

\section{Motivation}
\dictum[Ian S. Bruce -- Technology Editor of Sunday Herald]
{\flushright{}It’s Official: Video Games Are Bad for Your Brain. \cite{Bruce2002}}
Lately, the use of computer games to enhance learning has increased
significantly, but many educational games are very weak in terms of a good gaming
experience. They may be full with pedagogic principles, but they simply are not
entertaining or get boring quickly. On the other hand,
good games engage the player for many hours, which has worried some parents.
This lead to discussions
whether the influence of computer games on children is positive or negative
\cite{Southwell2004a}. The pipe dream of all teachers is to see
their pupils learning with the same motivation and engagement they play their
computer games at home.

I believe that this is not a dream, but creating good learning games can only
be successful if game developers, pedagogues and psychologists cooperate.
Educators as well as game developers have to combine their expertise to create games
which follow pedagogic principles and harvest the motivational factors of
cutting edge computer games. The most important thing is to optimize the game
over these two variables: fun (and as a direct result: motivation) and
pedagogical effectiveness. 

Non-linear games are very powerful for educational use, because they already
fulfill the factors Kiili has identified as most important in games: ``Games
should provide possibilities for reflectively exploring phenomena, testing
hypotheses and constructing objects.'' \cite{Kiili2005a}

%Educators from all over the world can only dream of what computer game
%developers have managed to accomplish: highly motivated children spend an
%awful lot of time with their products.
%Although many people have come to the conclusion that modern technology
%can enhance learning, the wide usage of such tools is still
%rare. There are a lot of tools available to support education, but the quality
%and effectiveness of such tools is too often very poor. There
%seems to be a very common problem in applied computer science: developers make
%software for educational purposes without enough knowledge off the pedagogic
%principles behind effective learning. This problem applies also to other
%fields of computer science, not only education. This results in tools
%with poor pedagogic background which fail to support traditional learning. A
%close collaboration with psychologists and pedagogues has to be established
%for every piece of software concerning education. 



\section{Problem Statement}
\label{problem_statement}
The problem with software in general and educational software or educational
games in special is that it always follows the principle of designers: Design
for the average user. Back in 1982, an experiment was conducted to study
whether this principle makes any sense or not \cite{Bailey1982}. In the
experiment, Bailey started with
4063 males selected randomly from the population of the USA. The scientist
measured the ten most characterizing physical features and reduced the sample
after each round. After the first round, only 1055 of the original 4063 males
(25.9{\%}) were close enough to the average standing height to continue the
experiment. When it came to the tenth round, only 2 subjects were left in the
sample and even they got eliminated in this last round. This shows that design
for the average is maybe good in theory, but in reality it does not make
much sense.

The trend in educational software (games and other tools) is developing towards an
adaptive system that focuses on the learner and adapts itself to
individual needs \cite{Shute2003b}. Classic computer games can be adapted quite
easily, if the system knows about the student model, the domain model and the
pedagogic model (see figure \ref{adaptive_game}). The adaptation mostly focuses on
changing the storyline and thus the presentation of the different learning
objects. Non-linear games cannot be adapted that easily though, because usually
there is no storyline that can be altered. Some non-linear games are task
based, which means that there are some tasks that can be fulfilled during the game,
without removing the exploration factor from the game. During the initial
research, some questions arose which had to be resolved in order to define what
intelligent support for non-linear serious games actually is:
\begin{itemize}
\item How can the computer detect if there was a problem during the game, if there
are no explicit tests?
\item Are the actions of the learner enough information to understand his
behavior, at least to a certain degree?
\item Can the game be adapted to the learner without taking the non-linear
element out of it? 
%~\item How can be we decide from the interactions with the game only, if the
%~learner has a problem with the current task?
%~\item When is the appropriate moment to give active support?
\end{itemize}
Summarized, the problem is to provide a good learning experience. In order to
do so, the above questions have to be examined closely and answers need to be
found to understand what the learner does in the game and support them
during their activities.
%~This thesis tries to get a grasp of the problem by analyzing and organizing the structure of
%~the application domain and provide an exemplary solution for the problem.

\begin{figure}
    \centering
    \includegraphics[width=0.5\textwidth]{diagrams/adaptive_game_uml.pdf}
    \caption[Components of an Adaptive Game (UML class diagram)]
    {Components of an Adaptive Game (UML class diagram)}
    \label{adaptive_game}
\end{figure}

%~My focus lies on educational games, to be
%~more specific on non-linear games. In order to adapt a game to the
%~player (learner), the game needs to be aware of the domain knowledge related
%~to the topic to be taught and of course it needs to understand the learner
%~itself. 
%~The key to good educational software is adaptability. The game has to adapt
%~itself to the learner and focus on the special needs of every single child.

%~The field of Game Based Learning (GBL) can be partitioned into two
%~big parts: linear and non-linear games. Linear
%~games have a clear plot and the learner follows the plot to reach
%~higher levels of knowledge. Non-linear games instead, give the learner
%~full freedom to explore a digital world. The learner becomes the explorer and
%~is immersed in the game.In non-linear games the learner is more a explorer and
%~learns things during the exploration. One of the most used examples
%~are physics games like {}``Phun'' or {}``Physics'' where you can
%~create objects and see manipulate them following physical rules (spin,
%~move, drop, explode, ...). I would like to focus on non-linear games
%~because for classic serious games there already exist a lot of solutions
%~for the problems I am about to address.

%~Non-linear games usually don't check what knowledge or skills the
%~learner acquired. This makes it difficult to adapt the game itself
%~or to help the learner with a topic.

%~The game should understand what the user and learner is about to do
%~and why he wants to do it. This is exactly what teachers do or at
%~least should do: anticipate the direction where the student is going
%~and guide him if he follows wrong leads. Most of the serious games
%~on the market cannot do that. They may have several plots which can
%~be adapted and lessons can be repeated, but what 

%~The problem with most serious games is that they do not adapt 

%~Current educational games capture the learning progress of the gamers
%~and learners with challenges (tests) the learners must face in order
%~to reach a higher level of the game or to fulfill the game. In non-linear
%~games it's not so easy to track what the learner already learned.


\section{Outline}
This section summarizes how this work is structured and what each chapter contains.
After the introductory first chapter, the next chapter provides a short summary
of several pedagogic principles behind the work: motivation, adaptation and
scaffolding support.

The third chapter presents some related projects which had a big influence on this
work. The projects presented are: ELEKTRA, MAGADI and weMakeWords. The chapter should
introduce the reader to a promising field of research which concentrates on the most
precious thing we have: education, the future of our children and our world. The presented
projects are intended as an appetizer for further reading.

After the related pedagogical principles and related projects have been presented, chapter
four contains two scenarios which show the need for better software tools to enhance
learning. The first part of chapter four presents a problem scenario (as-is situation)
and the second part of the chapter contains a visionary scenario which shows the
possibilities for improvement.

After the problems have been described, chapter five presents a solution to those problems.
In the fifth chapter I propose a system to address the problems of current educational
software systems, especially games. The chapter contains the functional and
non-functional requirements of the proposed system, use-cases for the system and
an exemplary flow of events.

Chapter six analyzes the application domain. The analysis helps the reader to understand
some of the decisions taken in later chapters. The concepts discussed in the analysis
chapter face the research questions presented at the end of the problem statement
\ref{problem_statement}.

The system design in chapter seven concentrates on the technical aspects of the
proposed system. Design goals are presented and the system is divided into major
subsystems. The chapter also describes the important aspects of the persistance
of some objects and presents a global flow through the subsystems identified in the
chapter.

Chapter eight covers the object design, where an extract of the complete
object model will be presented. The focus of the extracted section lies on
problem detection. After the object design, the last chapter covers the
conclusion and the future work parts.
