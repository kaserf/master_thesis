\chapter{Object Design}
\label{object_design}

\dictum[Bertrand Russell -- British philosopher]
{\flushright{}The greatest challenge to any thinker is stating the problem in a way that will allow a solution.}

This chapter covers the object design. Due to the limited time of the thesis,
only the most important part will be presented in this chapter. The presented
models are mainly extracted from the \textbf{Problem Detector} subsystem, which
represents the focus of this work. The first section will explain the
details of recognizing interaction patterns which are a essential part of
problem detection. The second section goes one step further and shows details
about how problems are actually detected.

\section{Matching Interactions to InteractionPatterns}

\textbf{InteractionPatterns} are a series of interactions which can be defined either
by hand or during the initial learning period of the framework. The first idea was to 
use regular expressions to define patterns. This is a simple concept which should do
the trick for the moment, but in future more advanced languages could be put in place.
The \textbf{InteractionPatterns} are stored in the \textbf{InteractionPatternStore} (which should
persist) and can be modeled as finite state machines, because they can be described
by regular expressions. When a new \textbf{Interaction} is classified and stored in the
\textbf{InteractionStore} by the \textbf{Action Monitor} subsystem, each
\textbf{InteractionPattern} gets notified (observer pattern).
The \textbf{InteractionPatternStore} knows which patterns exist and
instantiates the patterns at the startup of the system. The
\textbf{InteractionPatternStore} has to keep the information about the
patterns persistent, not in which state they are but which of them
exist. Each pattern starts with the default initial state of the state machine.

The \textbf{InteractionPattern} has to check if the newly classified \textbf{Interaction}
is a valid ``input character'' for the current state of the state machine.
If the \textbf{Interaction} is valid and leads to a transition, the state of
the pattern changes. Otherwise the state machine resets to the initial state. Once the state machine
reaches a final state, the observers which implement the
\textbf{PatternObserver} interface need to be notified
about the occurrence of a pattern. The
\textbf{PatternMatcher}, which represents the facade for the pattern matching,
does nothing else (for now) then relay the registration requests from observers to the
concrete \textbf{InteractionPattern} classes.
The \textbf{PatternMatcher} needs to store in a map, which observer is registered
to which \textbf{InteractionPattern}, to unregister the observers if needed.

\begin{figure}
    \centering
    \includegraphics[width=\textwidth]{diagrams/InteractionPatterns.pdf}
    \caption[Matching Interactions to InteractionPatterns (UML class diagram)]
    {Matching Interactions to InteractionPatterns (UML class diagram)}
    \label{matching_interactions}
\end{figure}

Figure \ref{matching_interactions} shows the object design of this part of the
framework. The classes shown in figure \ref{matching_interactions} are all part of
the \textbf{Problem Detector} subsystem, except the
\textbf{InteractionObserver} interface and the \textbf{ActionMonitorFacade},
which represent the interface to the \textbf{Action Monitor} subsystem.

\section{Detecting a Problem}
This section covers the part where actually the problems are detected. Until
now, we only handled abstract interactions and interaction patterns. To make
the problem detection as agile as possible, the framework uses the strategy
pattern to keep the various problem detection strategies exchangeable.

In figure \ref{problem_detection} the classes responsible for detecting a
problem are shown. The \textbf{Problem} class gets notified by the
\textbf{PatternMatcher} about any \textbf{InteractionPatterns} which matched
the current sequence of \textbf{Interactions} performed by the learner. When
the \textbf{Problem} gets notified about a new pattern, it executes the
current strategy which evaluates the context. The context of a
\textbf{Problem} are the \textbf{InteractionPatterns} which are related to the
\textbf{Problem}. A problem can be manifested in many different
\textbf{InteractionPatterns} and thus needs to keep a list of the
patterns which express the \textbf{Problem}.

\begin{figure}
    \centering
    \includegraphics[width=\textwidth]{diagrams/problemDetection.pdf}
    \caption[Problem Detection with Strategy Pattern (UML class diagram)]
    {Problem Detection with Strategy Pattern (UML class diagram)}
    \label{problem_detection}
\end{figure}

The strategies which implement the interface \textbf{ProblemStrategy}, need to
implement the method which takes the context as argument and produces a
boolean output which tells the caller, whether the actual context triggered the
problem or not. The strategy which is presented in figure
\ref{problem_detection} is an example strategy which rates each of the related
patterns with a probability factor. The probability factor determines to which
degree this pattern manifests the problem. The \verb=execute()= method sums up
the related patterns which have been detected recently. If the sum of the
probability values exceeds a limit, which can be defined in the \verb=trigger=
field, the method evaluates to true.
