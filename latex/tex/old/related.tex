\chapter{Related Work}
\dictum[Oliver Wendell Holmes -- American physician, professor, lecturer and author]
{\flushright{}People do not quit playing because they grow old;\\
they grow old because they quit playing.}

This chapter presents some projects related to the work done in this
thesis. This work is definitely inspired by the following projects and the ideas
behind the projects. The first project is ELEKTRA which is an interdisciplinary
project funded by the EU to develop a cutting edge learning game. The second
section presents MAGADI, a promising approach for an on-line blended learning
framework. Last but not least there is a short section about weMakeWords, which
is a very promising iPhone application with the potential to revolutionize for computer-based education.

\section{ELEKTRA}
The ELEKTRA project tries to combine the strengths
of cutting edge computer game technology and pedagogic principles which have
been used effectively in traditional learning. The team consists of
researchers and specialists from the fields of cognitive science,
neuroscience, pedagogy, game design and game development. This
interdisciplinary constellation of researchers developed a learning game which can
keep up with the standards of modern computer games and incorporates efficient
teaching methods.

\begin{figure}
    \centering
    \includegraphics[width=0.5\textwidth]{images/elektra.jpeg}
    \caption[The Logo of the ELEKTRA Project]
    {The Logo of the ELEKTRA Project}
    \label{elektra}
\end{figure}

ELEKTRA stands for Enhanced Learning Experience and Knowledge Transfer and is
a two-year research programme funded by the EU (under the Sixth Framework
Programme in the IST priority) \cite{Kickmeier-Rust2006}. The game, which was developed during the course
of the project, focuses on the school curricula and provides a virtual
environment where some of the experiments presented in class can be ``played''. The goal was to
develop a game which engages learners during their free time with topics
discussed in class.

The researchers faced many questions and
dug into a lot of interesting topics. Some of the topics discussed are: the gender effect
of the speaker in narrated animations \cite{Linek2006}, game-based learning
and meta cognition \cite{Castaigne} and the ontology model of the learner to
provide a learner-centric learning experience \cite{Kickmeier-Rust2008a}.

%~\begin{itemize}
%~\item project structure, goals, principles that can be used
%~\end{itemize}

\section{MAGADI}
MAGADI is the name of a lake in Kenya and also the name of a newly developed
framework which focuses on on-line blended learning. According to Alvarez, the
main modules of an Intelligent Tutoring System (ITS) are: the domain model, the student
model, the pedagogic module and the communication interface \cite{Alvarez2007}. Alvarez
criticizes that this approach misses one crucial part of education:
adaptation. To solve this problem, Alvarez et al. propose MAGADI, a framework
which supports blended learning. Blended learning tries to combine
face-to-face instruction with computer based methods \cite{Graham2004a}. The framework they
proposed is a ``domain independent, open and adaptive learning platform whose
adaptation style can be anytime modified by the teacher'' \cite{Alvarez2007}.

The important work which has been conducted with MAGADI is mostly in terms of
a good representation for the learner, the domain and the pedagogical model.
MAGADI tries to blur the line between on-line and off-line learning by
providing methods to integrate the off-line work into the system, so it can be
used by the educators in on-line assessments and helps to find the right
lessons for the learner during an on-line session.
%~\begin{itemize}
%~\item models, databases, learner represenation and especially domain representation
%~\end{itemize}

%~\section{Barwin}
%~The Barwin framework 
%~\begin{itemize}
%~\item models, use cases, analogies, learner centric framework
%~\end{itemize}

\section{weMakeWords}
During the course of a project at university, a literacy acquisition iPhone application
was developed. The goal of the software was to teach Chinese symbols to German
speaking children. The targeted audience were children between 4 and 8 years of age \cite{Ismailovic2009}. The
project was fully in the hands of students, who worked together with a child
psychologist who provided his insight on the topics related to literacy
acquisition of young children. The developed software was then evaluated with
children and the results were stunning. After only one hour of play, the
children were already able to recognize Chinese symbols correctly and also
reproduce some of them, which was initially completely out of scope for the
application.

The special focus of the iPhone application was on collaboration. The children
should play the game together, in one room not distributed and connected over
the Internet. Another goal of the project was to develop an adaptive game
which modifies itself if the game recognizes some problems during the play.
The first approach the students chose was to make the game harder and more challenging if
the learner is fast and learns easily. If the learner is not so fast though
and has problems learning the items, the game makes the tasks easier and not
so frustrating for the learner. 
%~\begin{itemize}
%~\item analogies, possible use of ped. agents or similar technologies, models
%~(especially strategy patterns)
%~\end{itemize}
