\chapter{Pedagogical Background}

\dictum[Marc Prensky -- founder and CEO of Games2train]
{\flushright{}Today's students are no longer the people our educational system was designed to teach.}

In this chapter, some findings from the fields of psychology and
pedagogics will be presented. These findings explain the necessity of intelligent support in
non-linear games and deliver insight into the techniques that should be used
in state of the art serious games.

\section{Achievement Motivation}
\label{achievement_motivation}
Motivation and self-efficacy are important factors for education and learning
\cite{Pintrich1990a}.
This is the conclusion of several studies conducted by some of the most
influential psychologists of the last century like Bandura \cite{Bandura1977a, Bandura1991a}
and Zimmerman \cite{Zimmerman1990a, Zimmerman1992a}.

Achievement motivation (AM) is a sub-part of the motivational studies conducted by
Bandura, Zimmerman and many more. AM consists of several theories which have
been summarized in \cite{Blanchard2004b}. These theories appear throughout the
field of psychology, not only in the study of education but also in social
psychology and others. The following list contains some of the most important
theories of achievement motivation:

\begin{description}
\item[Attribution Theory] is concerned with the impact
of different events on the personality of a person, especially those events the
person itself is responsible for. Outcomes of the actions a person takes can be
attributed internally or externally. Usually, people attribute success
internally (dispositional) and failure externally (situational). The most
famous example is about an exam at school or university: if a student
passes the exam, they usually attribute the success internally by
saying (or thinking) ``I learned so much for the exam
and was well prepared, that is why I passed!''. On the other hand, if a
student fails to pass the test they will attribute the failure externally: ``The
exam was harder then usual, the professor is to blame for the questions which were too difficult!''.
\item[Controls Theories] try to describe what impact the feeling of
self-control (control of the own achievement) has. In education it means: is
there  a positive correlation between self-control and learning outcome?
\item[Intrinsic Motivation Theories] distinguish between activities carried
out ``for their own sake'' \cite{Blanchard2004b} and activities with an
external reward (extrinsic). According to Eccles et al., the difference between
intrinsic and extrinsic motivation ``is assumed to be fundamental throughout
the motivation literature'' \cite{Blanchard2004b}. Blanchard states that the
self-determination theory of Ryan and Deci \cite{Ryan2000a} is probably the
most important intrinsic motivation theory.
\item[Theories of Self-Regulation] are theories researching how people
regulate their actions and their behavior in order to fulfill a task.
Self-observation, self-judgment and self-reaction are the processes identified
by Zimmerman \cite{Zimmerman1989a}.
\item[Academic Help Seeking] is a theory which emphasizes the importance of
``when'' and ``how much'' help is provided. Providing help in situations the
learner has never faced before can result in work-avoidant behavior, because
the learner is not sufficiently challenged.
\end{description}

\subsection{Most Important AM Factors According to Blanchard:}
\begin{description}
\item[Individual Goals] According to Eccles et al. \cite{Eccles1998}, there
are three different types of goals: ego-involved goals (aim at a positive
image through a good evaluation), task-involved goals (solve a task to aqcuire
competencies) and work-avoiding goals (minimize workload).
\item[Social Environment] The social environment has a strong influence on
self-belief, which is directly related to motivation. Especially ego-involved
goals are connected to the social environment, because the people who measure
and evaluate the performance of the learner are usually parents, peers or
teachers.
\item[Emotions] Emotions and motivation influence each other mutually.
Positive emotions can foster motivation in learning activities and negative
emotions cause the learner to lose interest and motivation. The
motivation on the other hand has also some influence over the emotion.
\item[Intrinsic Interest for an Activity] Intrinsic motivation means that the
learner is not motivated by external rewards. Blanchard states that there are
individual differences in intrinsic motivation of a learner: one learner may
show high interest in very hard and challenging tasks, the motivation of
another student may be curiosity and a third will search for tasks to enhance
its competencies.
\item[Self-Beliefs] Self-efficacy and expected outcome are some of the many
self-beliefs identified by Bandura \cite{Bandura1977a}. Self-beliefs describe
how much the learner believes in a positive outcome of the learning activity,
and are an important influence in reaching a goal: A learner who thinks that
the expected outcomes of an activity are low may not even try to perform the
activity.
\end{description}

\section{Adaptation Approaches}
\label{adaptation}
In the history of e-Learning various approaches to adapt learning have been
researched and categorized. The four main approaches according to
M\"{o}dritscher et al. \cite{Modritscher2004} are: the macro-adaptive approach, the
micro-adaptive approach, the aptitude-treatment interaction approach and
the constructivistic-collaborative approach. The following is a short summary of
these approaches, mostly from M\"{o}dritscher's work in \cite{Modritscher2004}
and \cite{Modritscher2006a}.

\begin{figure}
    \centering
    \includegraphics[width=\textwidth]{diagrams/adaptation.pdf}
    \caption[Different Adaptation Approaches (UML class diagram)]
    {Different Adaptation Approaches (UML class diagram)}
\end{figure}

\subsection{The Macro-Adaptive Approach}
Macro-adaptation mostly involves the change of learning objectives, levels of detail, delivery
system, etc \cite{Modritscher2004}. According to Corno et al. \cite{Corno1986}
the adaptation depends on the learning objectives and student aptitudes. It
makes a difference if the task should teach a new skill or compensate eventual
shortcomings. Student aptitudes inklude learning styles, motivation, previous
knowledge, personality etc. The macro-adaptive approach is mostly about the
sequencing of learning objects in respect to some of the factors named above
and can thus be implemented with the sequencing rules of SCORM (Sharable
Content Object Reference Model) \cite{Modritscher2006a}.

\subsection{The Micro-Adaptive Approach}
Micro-adaptation adapts instruction to the students needs during a task and
not only during a whole course. This makes it necessary to monitor the student
in-depth during a task to understand the behavior and act with respect to his
performance. The point is to use on-task measures instead of
pre-task measures other approaches use: it collects more detailed and more
dynamic information about the learner than pre-task measures do. The
information can then be used to adapt the instructions and provide a tailored
learning experience.

Existing models like the mathematical model, the trajectory model, the
Bayesian model, etc. allow micro adaptation to a series of instructional
variables such as the amount of content or the sequence of the content
\cite{Modritscher2004}. A promising implementation of the Bayesian model in
adaptive instruction has been presented in \cite{Rothen1978a}.

Micro-adaptation is divided into two processes: first the diagnostic process
which assesses the learners characteristics and properties of the task itself.
Useful information contains the prior knowledge, preferences, aptitudes
and content structure, difficulty level, conceptual model, etc.
\cite{Modritscher2004}. The second process is the prescriptive process
where the real adaptation happens. The goal is to optimize the interaction
between the learner and the task, with respect to learners' aptitudes and recent
performance (see Figure \ref{processes}).

%~According to M\"{o}dritscher, another important aspect of micro-adaptive
%~instruction is the response sensitivity.
Recent technological improvements allow to
collect even more data with technologies like eye-trackers or facial
recognition. This information leads to a more detailed
knowledge about the learner. It goes so far that researchers can measure and
categorize the emotional response of a student interacting with e-Learning software.
State of the art face recognition and eye tracker software can successfully
detect whether the student is motivated, bored, confused, etc. Knowing about
the emotional response of the learner is important, because the system can react
accordingly. An example: the system recognizes high frustration of a student
and adapts the lessons in order to make them easier and not so frustrating for
the student. When the negative emotions decrease, the system can increase the
difficulty of the lessons again. Emotions affect the motivation and poor
motivation results in a poor learning session \cite{Blanchard2004b, Blanchard2009,
DeVicente2003, D'Mello2007b}.

The last important factor in micro-adaptation is interactive communication.
The need for a well structured communications model, with special focus on the
interactions between the learner and the system (e.g. a virtual tutor), leads to
a model with two different communication channels: the teaching channel to provide content
and the assessment channel to monitor the learning process
\cite{Modritscher2004}. The two separate processes are highly correlated to
the two processes of micro-adaptation: the diagnostic and the prescriptive
process (see Figure \ref{processes}).

\begin{figure}
    \centering
    \includegraphics[width=\textwidth]{diagrams/micro_processes.pdf}
    \caption[The Two Processes of Micro-Adaptation (UML activity diagram)]
    {The Two Processes of Micro-Adaptation (UML activity diagram)}
    \label{processes}
\end{figure}

The micro-adaptive approach is very similar to what researchers developed as the
Experience Sampling Method (ESM). In experience sampling, participants of a
study are asked repeatedly over the time of a study to write down their
feelings and the actual situation. Proponents of this method argue that it
gives researchers the possibility to better understand the behavior, because
it allows to induce the behavior from the sampling data. Other methods deduce the
behavior from post-task measured data. The validity
and reliability of this method has been studied in the work of Mihaly
Csikszentmihalyi and Reed Larson \cite{Csikszentmihalyi1987a}.

\subsection{The Aptitude-Treatment Interaction Approach}
This approach adapts the instruction according to the students aptitude and
special characteristics. Aptitude Treatment Interaction (ATI) suggests the
usage of different types of instruction and different types of media for each
student. Important factors which determine the type of instruction and the
type of media are according to M\"{o}dritscher: intellectual abilities,
cognitive styles, learning styles, prior knowledge, anxiety, achievement
motivation and self-efficacy \cite{Modritscher2004}.

The special importance for e-Learning lies in the number of students that
usually should learn with e-Learning software. In school classes the size is
limited, because one teacher cannot teach more then 20-30 people.\footnote{In fact this
number should be much lower, but there are simply not enough teachers to
realize it.} In e-Learning the number of students who use a system is not
limited at all, thus it should support a variety of different learning styles,
instruction methods and media representations.

An important aspect of ATI is the user's control over his own learning
process. According to Snow, there exist three levels of control: complete
independence, partial control within a given task and fixed tasks with control
of pace \cite{Snow1980}. Studies have shown that students with
little prior knowledge should have limited control. This means the success of
the learners level of control is strongly dependent on his aptitudes
\cite{Modritscher2004}.

\subsection{The Constructivistic-Collaborative Approach}
The last approach addresses something that has been criticized very often by
e-Learning opponents: the lack of collaboration in e-Learning and the missing
social interaction. This approach follows the constructivistic learning theory
with special focus on collaboration. In this theory the learner plays an
important and active role in the learning process. M\"{o}dritscher quotes
\cite{Akhras2000} that ``constructivistic learning may benefit from a system's
intelligence including mechanisms of knowledge representation, reasoning and
decision making'' \cite{Modritscher2004}.

This approach features new pedagogic theories and tries to
understand how knowledge acquisition works. The collaborative aspect of
learning has been praised in the last years by pedagogues and has also been
proven to be very effective. Soller identified in her work \cite{Soller2001}
five characteristics of effective collaboration learning: participation,
social behavior, performance analysis, group processing and conversation
skills, and primitive interaction. With these characteristics in mind, the way
is open to take the step from individual e-Learning software to collaborative
e-Learning experiences.

\section{Scaffolding}
\label{scaffolding}
The principle of scaffolding is simple: a tutor, teacher or peer should take
over parts of the task which are too complex for the learner to carry out
alone from the beginning. This means the learner initially has to do only the tasks
of which they have prior experience, and that over time the load increases until
the learner is capable of doing all the work without intervention. The idea
behind scaffolding is to confront the learner
with big and complex tasks, too complex to be solved alone. The complexity of
such tasks is very engaging, argue supporters of the scaffolding principle.

Scaffolding is based on Vygotsky's zone of proximal development (ZPD), which is one
of the leading theories in constructivist learning. The  ZPD describes the
difference between the problem solving skills of a child during an actual task
without guidance and the potential level a child could reach with the support
of adult guidance and interaction with more experienced peers. According to
ZPD a interpersonal level of learning is needed first, before the things
a child learned in interaction with others can be internalized \cite{Roehler1997a}.

Brian Reiser presented a study where he researched how scaffolding can be used in
software tools. He states in his work about scaffolding in software: ``Software tools can
help structure the learning task, guiding learners through key components and
supporting their planning and performance'' \cite{Reiser2004a}. His work shows
many promising ideas for scaffolding with software tools.

%~\begin{itemize}
%~\item get the good stuff summarized from \cite{Blanchard2004b}
%~\item pedagogical agents (?)
%~
%~\item scaffolding (?)
%~\item goal based scenarios
%~\item experience learning in an open environment 
%~\end{itemize}
