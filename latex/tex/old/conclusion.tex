\chapter{Future Work}
\label{future_work}
\dictum[Martin Fowler -- an author, speaker and consultant on software development]
{\flushright{}Any fool can write code that a computer can understand. Good programmers write code that humans can understand.}

Reaching the goal of developing an intelligent framework which understands the
behavior of the learner still needs a lot of work, but with this work we are one
step closer. This chapter mentiones some thoughts how this work should be carried
on, concluding with a small excursus to a project which could be a perfect
platform for the framework (see \ref{olpc}: One Laptop Per Child).

\section{Extending the Framework}
This work describes a conceptual framework for the problem detection in non-linear
serious games. The prototypical implementation covers only a part of the described
framework, due to time restrictions. To provide a more realistic implementation,
new strategies for dealing with problems can be added.

\section{Interaction Patterns}
In this work, the interaction patterns are described with regular expressions (see
chapter \ref{object_design}). In future work, it should be evaluated if other
formal languages are more appropriate to describe the patterns which can occur during
the process of learning. Methods need to be implemented to create, load and store
interaction patterns at runtime.

\section{Artificial Intelligence}
As already suggested in \ref{core_framework}, the \textbf{Problem Detector} subsystem
would be the perfect place to introduce artificial intelligence. In this subsystem,
an artificial intelligence component could be used to improve problem detection.
Extensive research is needed to determine how artificial intelligence can be used for problem
detection.
%~During this work, a language was mentioned to describe interaction patterns
%~(see chapter \ref{object_design}). A closer investigation is needed to define
%~this language and develop a method to create, load and store interaction
%~patterns at runtime. The language needs to express every possible patterns
%~which can occur during the process of learning.

\section{Games using the Framework}
The framework was designed using examples from games which already exist.
Once the framework is completed, a game needs to be developed or extended, which uses the
framework and implements a concrete pedagogical agent. The game should be a
non-linear serious game where the actions are clear and interactions, patterns
and problems can easily be defined. Defining these structures requires a close
collaboration with a psychologist and maybe a small experiment is needed to
pick out a good set of interactions, patterns and problems.

\section{Evaluation}
After the full framework is implemented and a game has been developed, the
system should be evaluated and data about the efficiency of the
framework needs to be collected. Empirical studies can prove or disprove that the framework
is able to provide a deeper and better learning experience. After the
evaluation, the framework could be refined to feature a better set of
interactions, interaction patterns, problems and support strategies.

\section{One Laptop Per Child}
\label{olpc}
This section is dedicated to the One Laptop Per Child
project\footnote{www.laptop.org}. The OLPC project was started at MIT Media
Lab with the goal to develop a laptop which costs only \${100} and is robust
enough to operate in the most remote places on earth. What the OLPC tries to
achieve is to provide the children of this world with the tools to learn and
to share their ideas. The device which was developed to suit the needs of the
project was called XO and shipped to many developing countries. The XO would be
the perfect target device for this framework, because the platform was
explicitly designed to teach and the framework could reach many children
playing and learning with a XO. At the moment, the mass production of the
second version of the laptop is starting: the XO1.5. The XO1.5 comes with the same body as the
original XO but is much faster. The original XO was fast enough for developing
countries, but it seemed too slow for the needs of other countries. With the
XO1.5 this barrier is broken and the hope is big that we will see many devices also around
Europe in the near future.

%~\section{Future Work}
\label{future_work}

\begin{itemize}
\item more psychological backends
\item ***adapt categories***

\begin{itemize}
\item not only scaffolding, maybe there is a need of domain specific instruction
principles
\end{itemize}
\item personal agents (gender based?)

\begin{itemize}
\item agents can be given a kind of personality: the child defines what
the agent looks like, what his hobbies are, ...
\item agents modify themselfes regarding to the information the child provided:
gender, age, interests
\end{itemize}
\item evaluation

\begin{itemize}
\item field study
\item proof that the system works and improves learning
\end{itemize}
\item ui experiments

\begin{itemize}
\item usability of agents that are sometimes visible and sometimes not
\end{itemize}
\end{itemize}

%~\section{Evaluation}

\begin{itemize}
\item only if there is time enough, else this belongs to future work
\end{itemize}

%~\section{Prototypical Implementation}

\begin{itemize}
\item done / to do
\item stubs
\item problems that appeared
\end{itemize}

%~\section{Backlog}

connection with peers (how can we detect a problem)


\chapter{Conclusion}
\dictum[Bertholdt Brecht -- German poet]
{\flushright{}...it is simplicity that is difficult to make.}
What was proposed in this work is a framework which watches the learner
during a non-linear serious game and detects problems during the learning
process. The overall idea and vision behind this work is to develop a
framework which understands the behavior of the learner.

The framework was designed and some of the key functionalities were addressed
in detail. The work should lay the foundation for future research and serve as
a starting point for other theses. The work presented a solution for the
complex relationships between actions, interactions and problems. This
solution needs to be tested and adapted, if the results show that it is not
yet suitable for the application domain.
%~The vision behind this work is to develop a framework which fully understands
%~the learner. I think the framework presented in this thesis lies a good
%~foundation on how this problem can be solved. Anyways I have to accept that
%~the framework is not perfect and has several parts which need to
%~be completed in future work. This thesis should be an inspiration for future research and projects,
%~because I believe that we can reach the goal of a behavior aware learning
%~application.

The proposed framework shows how detecting problems of a learner
during the learning experience is possible. An evaluation is needed to decide
if the approach this work chose is valid or should be adapted. Detecting problems is the first step, but in
order to fully understand the learner more steps are needed. Thats the point where
the collaboration with psychologists is essential to define which parts of human
behavior are most important for learning.

I see a lot of potential in this field of research, because the
education of our children is the greatest good we have and there are several
ways how we can support it. We should work hard to
improve the learning experience we provide to our children. The One Laptop Per Child (OLPC)
project\footnote{www.laptop.org} is very ambitious about that: they work hard to bring education to the poorest children on this
planet, because they believe that it is the only way to solve the problems of
tomorrow.
%~
%~\begin{itemize}
%~\item befor or after future work?
%~\item my opinion about the system and the work that needs to be done
%~\item possible advantages in schools everyday
%~\end{itemize}
