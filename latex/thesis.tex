\documentclass[12pt,a4paper,headsepline,footsepline,DIV13,BCOR12mm]{scrbook}
\usepackage[english,ngerman]{babel}
\usepackage[T1]{fontenc}
\usepackage[latin1]{inputenc}
\usepackage{scrpage2}
\pagestyle{scrheadings}
\usepackage{setspace}
\usepackage[font=singlespacing]{caption}
\usepackage{graphicx}
\usepackage{subfigure}
\usepackage{listings}
\usepackage{comment}

%create pdf hyperlinks with a color box (will not be printed)
\usepackage{hyperref}
\hypersetup{colorlinks=false}

% todo and review commands
\usepackage{soul}
\usepackage{color}
%\newcommand{\hl}[1]{\fcolorbox{red}{white}{#1}} %fcolorbox does not break lines properly
\newcommand{\todo}[1]{{\color{red}{\hl{TODO: #1...}}}}
\newcommand{\review}[0]{\todo{REVIEW}}
\newcommand{\comm}[1]{\textsuperscript{\bf \color{red}{\tiny [#1]}}}
\newcommand{\q}[0]{\comm{?}}
\newcommand{\s}[0]{\comm{*}}


\begin{document}

% begin preface
\selectlanguage{ngerman}
%%%%%%%%%%%%%%%%%%%%%%%%%%%%%%%%%%%%%%%%%%%%%%%%%%%%%%%%%%%%%%%%%%%%%%%%%%%%%%%%%%%%
%%% Title-Page
\thispagestyle{empty}

\begin{table}[h]
\centering
\begin{tabular}{ccc}
\includegraphics[width=.18\linewidth]{img/logos/lmu_logo} \hspace{0.7cm} &
\includegraphics[scale=0.18]{img/logos/Uni_Aug_Logo_Basis_pos_A}
\hspace{0.7cm} &
\includegraphics[scale=0.86]{img/logos/tum}
\end{tabular}
\end{table}

\vspace{8mm}
\begin{center}
{\Large
{\bfseries \scshape Institut f�r Software \& Systems Engineering}\\
Universit�tsstra�e 6a \hspace{0.25cm} D-86135 Augsburg\\
}
\end{center}

\vspace{1cm}
%title
\begin{center}
{\Huge \bfseries A flexible visual framework  \\
\vspace{0.25cm}
for debugging complex robotic systems}
\end{center}

\vspace{1.5cm}
%author
\begin{center}
{\Large Felix Kaser}
\end{center}

\vspace{1cm}
\begin{center}
{\Large \bfseries Masterarbeit im Elitestudiengang Software Engineering}
\end{center}

\vspace{1cm}
\begin{center}
\includegraphics[width=.4\linewidth]{img/logos/LogoSEengl}
\end{center}



%%%%%%%%%%%%%%%%%%%%%%%%%%%%%%%%%%%%%%%%%%%%%%%%%%%%%%%%%%%%%%%%%%%%%%%%%%%%%%%%%%%%
%%% Advisor-Page
%%%%%%%%%%%%%%%%%%%%%%%%%%%%%%%%%%%%%%%%%%%%%%%%%%%%%%%%%%%%%%%%%%%%%%%%%%%%%%%%%%%%
\newpage
\thispagestyle{empty}
\mbox{}
\newpage
\thispagestyle{empty}

\begin{table}[h]
\centering
\begin{tabular}{ccc}
\includegraphics[width=.18\linewidth]{img/logos/lmu_logo} \hspace{0.7cm} &
\includegraphics[scale=0.18]{img/logos/Uni_Aug_Logo_Basis_pos_A}
\hspace{0.7cm} &
\includegraphics[scale=0.86]{img/logos/tum}
\end{tabular}
\end{table}

\vspace{1cm}
\begin{center}
{\Large
{\bfseries \scshape Institut f�r Software \& Systems Engineering}\\
Universit�tsstra�e 6a \hspace{0.25cm} D-86135 Augsburg\\
}
\end{center}

\vspace{2.5cm}
%title
\begin{center}
{\Huge \bfseries A flexible visual framework  \\
\vspace{0.25cm}
for debugging complex robotic systems}
\end{center}

\vspace{1cm}
%author
\begin{center}
\begin{table}[h]
\centering
\begin{tabular}{ll}
Matrikelnummer: & 1174068 \\
Beginn der Arbeit: & 21.\ Juni 2012 \\ 
Abgabe der Arbeit: & XX.\ XXXX 2012 \\
Erstgutachter: & Prof.\ Dr.\ Wolfgang Reif \\
Zweitgutachter: & Prof.\ Dr.\ YYY YYYYYY \\
Betreuer: & M.Sc. Andreas Angerer \\
\end{tabular}
\end{table}
\end{center}

\vspace{1.25cm}
\begin{center}
\includegraphics[width=.4\linewidth]{img/logos/LogoSEengl}
\end{center}

%%%%%%%%%%%%%%%%%%%%%%%%%%%%%%%%%%%%%%%%%%%%%%%%%%%%%%%%%%%%%%%%%%%%%%%%%%%%%%%%%%%%
%%% Statement-Page
\newpage
\thispagestyle{empty}
\mbox{}
\newpage
\thispagestyle{empty}

\centerline{\bfseries ERKL�RUNG}

\vspace{5cm}
Hiermit versichere ich, dass ich diese Masterarbeit selbst�ndig verfasst habe.
Ich habe dazu keine anderen als die angegebenen Quellen und Hilfsmittel
verwendet.

\vspace{1cm}
\begin{flushleft}
%select german for formatting the date
\selectlanguage{ngerman}
Augsburg, den \today \hfill Felix Kaser
\end{flushleft}

\newpage
\thispagestyle{empty}
\mbox{}

 
\newpage

\renewcommand{\baselinestretch}{2} % zeilenabstand

%select english as language!
\selectlanguage{english}

\pagenumbering{roman}
\chapter*{Abstract}
Data collected during debugging is traditionally rendered as text. The special application field of robotics faces problems with this approach, since a typical robotic system constantly gathers and processes data from the surrounding environment through sensors. Robotic applications are also hard to interrupt during debugging, since robots generally don't run in a deterministic and suspendable environment. Developers of robotic applications are confronted with high amounts of data during debugging, which becomes hard to interpret if the data is represented as text and high amounts of data need to be interpreted at once. This thesis tries to solve this problem by introducing a visual debugging system to support debugging of robotic applications: It takes into account the special requirements for a debugging tool in a robotic development environment, especially the uninterrupted rendering of debugging data and the need for better visualization of data to support a faster interpretation of data. The goal of the developed system is to help developers understand the data during debugging more quickly and improve overall productivity during robot development. To achieve the goal a system was designed and developed where developers can choose how they want to visualize data collected during debugging of a robotic application. This work presents the system design and a prototypical implementation of the proposed system. Evaluating the hypothesis that the developed system can improve productivity during debugging is out of scope for this work but the developed tool provides a solid base for further evaluations.
\newpage
{\huge \todo{abstract deutsch????}}
\newpage
\chapter*{Acknowledgment}
I would like to thank everyone who supported in the last months either by providing advice and guidance or being a good friend and listen to my ideas and thoughts. Thank you.

\todo{thank bruce, andreas, ludwig, jamie, tim, etc}
%Special gratitude goes out to the persons that enabled me to work on this topic: The staff in the Software Engineering Elite Graduate Program for enabling me to write the thesis in New Zealand, Prof. Bruce MacDonald and his staff at the University of Auckland, New Zealand for the constant support and advice and last but not least my family for supporting me mentally and financially whenever I needed their support.
\todo{acknowledgement before abstract?}
\newpage
\chapter*{TODOs}
\todo{fix chapter name for appendix}
\todo{change layout of images to be in-text}
\todo{add diagram type? e.g. UML class}
\todo{semantics of communication diagram}
\newpage
\tableofcontents
\newpage
\listoffigures
\newpage
%\listoftables
\lstlistoflistings

\pagenumbering{arabic}
%end preface

\chapter{Introduction}

Current debugging tools for robotics are mostly based on debugging techniques and interfaces developed for traditional non-robotic applications. Those techniques and interfaces were developed specifically to suit the requirements of traditional systems: First they assume that a program can be interrupted in its execution, which in a robotic environment is not possible most of the time. Second the data handled in traditional applications is usually discrete and based on user input, as opposed to the sensory data a robot handles. The data a robot handles is in general closely related to the real world environment of the robot and comes from sensors that provide a continuous data stream of readings. Although the computing world has changed in recent years, most tools have not. Robot developers often still use tools which were developed under different circumstances and based on different requirements. The application field of robotics has special requirements for debugging tools which are often not met by current debugging tools.

Traditional debugging tools usually render data collected during debugging as text and it's the developer's task to interpret the data. In a suspendable environment the developer has as much time as they need to interpret and analyse the data. When debugging robotic applications the developer usually has a lot less time, because of the nature of the application: Robotic applications usually can not be interrupted in their execution, because robots generally don't run in a deterministic and suspendable environment \cite{Gumbley2009}. Interrupting a robotic application would destroy the continuity in which the sensors collect data, the environment of the robot would change substantially and thus the robot's behaviour would change, which is an example of the ``probe effect'' and makes it hard to reproduce a fault unless it has a single cause \cite{Gumbley2009}. It is necessary to collect data during debugging without interrupting the execution of the program. Although some technical solutions have been presented in recent years \cite{Gumbley2009}, many developers still rely on printf-style debugging or other logging mechanisms. This approach is much simpler and does not require external tools, but the source code must be modified. Source code modification can be a problem with so called "Heisenbugs", software faults that disappear because the observation affected the bug \cite{Grottke2005}. The data collected with print or logging statements is usually text-only, which requires the developer to constantly parse and interpret logging messages. Due to the large amount of data that is processed this often means developer consoles are filled with logging messages that often contain more complex content such as numeric data.

ROSDashboard, the tool presented in this work, aims to support the developer during debugging by visualizing data in a graphic way and thus eliminating the cognitive effort needed to parse and interpret text based logging messages. While most of the currently available visualization tools in robotics focus on spatial data to help understand the robot and the environment in which it runs \cite{Collett2010, Quigley2009}, rendering of abstract data is still uncommon. ROSDashboard provides a dashboard interface to robot developers, which they can populate with graphical widgets to visualize all kinds of data from the robot. The dashboard can be customized to display widgets according to the current robot hardware and development stage. It can be used to visualize data during debugging as well as monitor data during the normal execution of the robot. This means ROSDashboard is a tool that a) can be adapted to many different use cases and b) allows the developer to choose the widgets he or she thinks represent the data best, according to their mental model and the meaning of the data. The tool is based on ROS (Robot Operating System) which abstracts from specific robot hardware and takes care of inter process communication \cite{Quigley2009}.

\section{Problem Statement}
\label{problem_statement}
[\textbf{problems with robotic software, many different environments, not interruptable, ...}]
Debugging robotic systems can prove to be much more complicated then normal software systems. 

\section{ROS}
The Robot Operating System (ROS) is an Open Source framework for complex robotic systems. The first work on ROS was done as part of the STanford Artificial Intelligence Robot (STAIR) in 2007 \cite{Quigley2007}. The original software library was called \emph{Switchyard} and had been developed at Stanford. Later the library was refined and generalized to also suit the requirements of the Personal Robot Program at Willow Garage\footnote{www.willowgarage.com} \cite{Quigley2009}. The resulting general framework has been released as Open Source \cite{Quigley2009} and this section gives a short overview over the most important principles in ROS.

\begin{figure}[ht]
\centering
\subfigure[Stanford's STAIR]{
	\includegraphics[height=8cm]{img/stair}
}
\subfigure[Willow Garage's PR2]{
	\includegraphics[height=8cm]{img/pr2}
}
\end{figure}

ROS was built to abstract from the hardware of the robot and create modular robot software, which can run on different robots and on different machines. This makes it easier to write software for robots and distribute the work to different teams, each team focusing on one part of the robot. The modules in ROS are called nodes and several nodes executed together are called a stack. ROS packages bundle nodes and stacks and are used to make software modules available to other developers. Everyone can create their own package which can be indexed by ROS so that your software modules can be found, downloaded and used by other developers. There exist many packages, nodes and stacks for some of the most common problems in robotics (e.g. navigation, localization, joint movement, etc.) and can easily be re-used.

The communication between ROS nodes can be done asynchronously through a publish/subscribe mechanism and synchronously through services. Nodes can send messages by publishing a message on a topic and receive messages by subscribing to that topic. This mechanism is really flexible and decouples the sender from the receiver. A publisher node does not need to know if there are other nodes listening and vice versa. For synchronous communication and guaranteed delivery of messages, services can be invoked. The routing is established during runtime through the ROS core. The core of ROS was kept really slim and only contains the most essential parts of the framework (such as the inter node communication). ROS can run on several machines distributed in a network, the only restriction is that every node needs to know the address of the core (master node) in order to communicate with other nodes.

\review

A variety of tools have been built around the ROS core to facilitate the development of ROS nodes and robotic software in general. They help you to create packages, nodes and stacks and execute and debug them. The philosophy for those tools is to be small and do one job only but do it good. This results in a really robust tools similar to the toolchain available on Linux. The downside is a big variety of tools in the ROS ecosystem, which can be confusing to new developers. See Section~\ref{related_ros_tools} for a more detailed analysis of current ROS tools.

The latest stable version of the ROS framework was released in April 2012 (ROS Fuerte). Previous releases of stable versions have been in August 2011 (ROS Electric), March 2011 (ROS Diamondback), August 2010 (ROS C Turtle) and March 2010 (ROS Box Turtle). There are currently many institutions, companies and individuals involved in the ROS community, contributing in many different ways. This makes ROS a god target platform, since we want to reach as many developers as possible (see Section~\ref{availability_developers}).

\section{Goals (?)}
Maybe it would be good to explain the goals of this work in the introduction: 

\section{Outline}
Explain all the chapters and sections of this work.

\chapter{Debugging in Robotics}

[Write about the difficulties in robot debugging in general as a short introduction]

Debugging robotic applications is even more difficult then debugging normal systems. This is due to the nature of robotic applications: they run in a real environment and handle data that has been collected from the real world through sensors. Robotic applications often run on a distributed network and thus on many different processors at the same time.

The distributed, close to the real world and often real time constrained application field of robotics stresses currently available debugging tools.

Due to many different application scenarios in robotics and the diverse environment of available frameworks for robot development, many researchers and developers have built their own tools to support them during debugging \cite{Collett2010}. This has lead to a high amount of different debugging tools for different purposes.

The special application field of robotics faces problems with this approach, since a typical robotic system constantly gathers and processes data from the surrounding environment through sensors. Robotic applications are also hard to interrupt during debugging, since robots generally don't run in a deterministic and suspendable environment. Developers of robotic applications are confronted with high amounts of data during debugging, which becomes hard to interpret if the data is represented as text and high amounts of data need to be interpreted at once. This paper introduces a new tool to support debugging of robotic applications: It takes into account the special requirements for a debugging tool in a robotic development environment, especially the uninterrupted rendering of debugging data and the need for better visualization of data to support a faster interpretation of data. The tool's goal is to help developers understand the data during debugging more quickly and improve overall productivity during robot development. A graphical user interface was developed where developers can choose how they want to visualize data collected during debugging.

\section{Instrumented Debugging Tools}
This section introduces some of the most instrumented  debugging tools used for robotics. The tools presented in this section usually don't require source code modification and can be used for every robotic framework or middleware [check if thats true for tracepoints and gdb]. Although gdb is a highly sophisticated tool with many features, this section will focus only on the features that are most important for robotics.
\subsection{gdb}
[explain gdb tracepoints and gdbserver, introduce (gdb) tracepoints and their downside (postmortem).]

\subsection{Realtime Debugging Tracepoints}
[Luke Gumbleys work, tracepoint theory, adaptation in robotics, instrumentation]
\cite{Gumbley2009}

\section{Visual Debugging in ROS}
\label{debugging_ros}

%[summarize what ros is]
ROS (Robot Operating System) \cite{Quigley2009} is an Open Source framework for complex robotic systems which has grown significantly in the last years, has an active community backing the project and supports many of the currently available robots \cite{Foote2012}. It was developed to abstract from the hardware of the robot and make it easier to create modular robot software, which can run on different robots and on different machines. The modular approach makes development easier, because the work can be divided amongst different developers or development teams. This also allows the developer to change only small parts of a complex system, without the need to build and re-deploy the whole system.
The modules in ROS are called \emph{nodes} and several nodes executed together are called a \emph{stack}. ROS \emph{packages} bundle nodes and stacks and are used to make software modules available to other developers. Everyone can create their own package which can be indexed by ROS so that their software modules can be found, downloaded and used by other developers. There exist many packages, nodes and stacks with implementations of algorithms for some of the most common problems in robotics (e.g. navigation, localization, joint movement, etc.) and they can easily be (re-)used.

%** again, state the relevance to debugging **
The communication between ROS nodes is either asynchronous through a publish/subscribe mechanism or synchronous through services. Nodes can send messages by publishing a message on a topic and receive messages by subscribing to that topic. This mechanism is flexible and decouples the sender from the receiver. A publisher node does not need to know if there are other nodes listening and vice versa. For synchronous communication and guaranteed delivery of messages, services can be invoked. The routing is established during runtime through the ROS core. The communication between nodes is one of the main sources for debugging data when debugging a ROS application. The same communication framework is also used for the logging mechanism in ROS, which publishes messages to the special purpose topic \emph{/rosout}.

This section gives an overview of the ROS tools that can be used to monitor the ROS communication channels and thus debug ROS applications. ROS has both command line tools and tools with a graphical interface. Often there is a graphical interface based on the command line tool, in which case this section will focus on the graphical part, since it basically has the same functionality as the command line interface.

%[skip next paragraph]
%ROS comes with several tools to assist the developers during the development and debugging of a robot. The tools most relevant to this work are:
%\begin{description}[\setlabelwidth{rxconsole}]
%\item[rostopic] a command line tool to monitor topics and publish messages on topics.
%\item[rxplot] a graphical tool which plots data from one or more topic fields on a cartesian coordinate system.
%\item[rxconsole] displays logging messages that have been published on the special purpose topic \emph{/rosout}.
%\item[rviz] renders models of the robot in 3D and visualizes spatial data like point clouds, robot poses, trajectories, etc. \cite{Quigley2009}.
%\end{description}

\subsection{RViz}
RViz is a highly sophisticated graphical interface to render 3-dimensional data\footnote{http://www.ros.org/wiki/rviz}. The tool visualizes a 3d model of the robot an can visualize additional data like point clouds, robot poses, trajectories, etc. \cite{Quigley2009}. During the development of robot applications, it is hard to understand how the robot perceives his surroundings, because the robot is bound by sensor data which might not always collect information accurately enough. This leads to problems during the execution which are hard to identify, reproduce and investigate due to the indeterministic nature of robotic applications. RViz helps developers to understand how the robot sees the world and makes it easier to understand what caused a problem.

[screenshots from the wiki?]

RViz requires a 3d model with the exact proportions of the robot to visualize the robot and its movement correctly. The data visualized in RViz is mostly pre-defined and collected from known interfaces such as laser sensors, mapping algorithms and computer vision modules. Although it is possible to visualize arbitrary data in RViz, it is hard to find a good place to visualize the data, because the data does not have to be directly related to the real world, it can also be intermediate values from a computation algorithm.

[write about rviz plugins]

\subsection{rxconsole}
rxconsole is a graphical user interface to display messages that have been published on the special purpose topic \emph{/rosout}. The special purpose topic \emph{/rosout} is used by the ROS logging framework (see \ref{ros_logging}). rxconsole keeps a list of messages with additional information like the severity of the message, the node which published the message and the timestamp when the message was published. The tool also allows filtering by severity and text.

[screenshot?]

rxconsole is designed to only display text messages since the logging framework converts everything to a text message before publishing. Important information about the type of data is not preserved. rxconsole is thus a very simple tool to monitor logging messages, it does not offer any special visualization for data, which would be hard since the messages are text based and don't contain type information.

\subsection{rxbag}
In a complex robotic system there is usually a lot of communication happening at once. Trying to keep track of the different topics and messages is not an easy task and recording the messages for analysis at a later stage is necessary. rosbag is a tool to dump communication data from selected topics to a file for later use or to analyse the communication after the application has terminated. The data is stored in a so called bag-file, which contains all the messages published to selected topics during the runtime of the application. rosbag can also be used to play back the recorded messages in the same order and with the same time offset as they were recorded. This can be used to examine the behaviour of one particular subsystem without the need of running the full stack of subsystems in a real environment.

[example use of rosbag?]

The bag-file can also be used to examine a flow of events in the system after the execution has terminated. rxbag is a graphical tool to analyse bag-files, it visualizes the content of a bag-file on a timeline. The tool has controls to play back, pause and rewind the stream of messages, which allows to have a closer look at some events on the timeline. rxbag visualizes image messages as thumbnails on the timeline which makes it easier to understand what the robot was facing when the messages were recorded. Developers can use rxbag to inspect messages in more detail, since the raw messages were recorded and can be accessed from the tool.

[rxbag screenshot?]

While rxbag mostly focuses on replaying recorded messages from a bag file, it can also be used to visualize the data stream while it records it. This can be done with the command line option \emph{--record}. rxbag can be extended with more visualizations through a plugin API. The plugin API is still experimental and under development, but it should become more stable in future versions of ROS \cite{rxbag}.

[epiphany: difference between rosdashboard and rxbag: rxbag focuses on recorded data, but can also do live visualization. The biggest difference is that rxbag needs a static configuration (what topics should be recorded) and does not allow ad-hoc addition of new topics which might only be used for debugging reasons. This makes the configuration more time consuming because after adding a new logging statement the whole capturing process needs to be adapted and restarted to get the new data. rosdashboard can be left untouched, only a new widget needs to be added]

%rxbag does something similiar we do: visualize data. It visualizes data from a bag file: you can see image data in a picture-stream way, plot numerical data and look at raw messages. You can also write plugins for more visualizations. The downside is that it only operates on pre-collected data sets (bags) and not on live data. This would be interesting to explore further, to visualize data live and collect the data for later analysis (combining rxbag with rosdashboard). This would not be a big problem, because collecting the bag file while we are visualizing is not a problem.

\subsection{rxplot}
rxplot is a graphical tool which can plot values from topics on a Cartesian coordinate system. 

\subsection{RQT - ROS GUI}
\subsection{gdb and ROS}
\subsection{rxDeveloper}
[\textbf{outline, results of the survey, importance for this work}]
\cite{Muellers2012}

[Not really debugging, this might go somewhere else? Maybe not a full subsecion but only a couple of sentences that summarize the results and why it is important for this work]

\section{Graphical Debugging Systems}
\subsection{LabView}
[\textbf{work out the importance of 6.2. Non-Visual Theme: Front Panel (GUI) Setup}]
\cite{Whitley2001}
\subsection{Augmented Reality Debugging System}
\cite{Collett2010}
\subsection{Microsoft Robotics Studio}
\cite{Jackson2007}

\section{Logging Frameworks}

[probably unrelated to debugging in robotics? maybe have a section "other debugging tools?" or summarize this in the ROS logging as other logging frameworks]
\subsection{Android LogCat}
\subsection{Java Log4j}
\subsection{ROS Logging}



\chapter{Design of a Flexible Visual Debugging System/Tool/Framework?}
\label{visual_debugging_system}

Debugging in robotics has unique requirements which are often not met by traditional debugging systems. The special requirements lead to a number of different tools to support debugging in robotics. The previous chapter summarized some of those tools and issues of existing solutions were identified: a) The tools either focus on visualization of pre-defined spatial data or render data as text messages. b) The graphical interfaces are rather static and difficult to adapt to different usage scenarios.

This chapter introduces a new system for debugging in robotics, tailored to tackle the challanges developers face when debugging robotic systems. The first section states the hypothesis behind the ideas that have driven the development of the debugging system. Section two will state the goals of the work, which leads to specific requirements for this system. The system design section presents the design for a flexible visual debugging system.

[----> where should I introduce the word dashboard and widget? goals? user interface?]

%Write about the general requirements for a visual debugging system, rosdashboard would be one possible implementation. This basically could explain most of rosdashboard, apart from the ros specific part which is: topic subscription setup (gui), topic introspection to wait for message class types, topic subscription (code) and calling the value update hook. Saving to file saves the topic configuration which is ros specific as well.

%Point out that ROS was targeted early on and affected many decisions, but the requirements and the system design is applicable in general. [Andreas: not sure if this is the right place, might be better in the introduction, since it explains the dominance of ROS tools in related work / debugging in robotics]

%--> Generalize design decisions, if they apply for other robot frameworks as well.

%%%% HYPOTHESIS %%%%
\section{Hypothesis}
A possible reason why many developers still use print or logging statments to debug their programs is because these methods can be immideately used when required. Developers don't need to undergo a lenghty setup and configuration process before they can start debugging and they don't need to gain extensive knowledge about the methodology and tools either. The overhead to set up and configure a debugging system should be reduced to make the system easy to use without extensive knowledge about the debugging methodology and tool.

The main issue with print or logging statements is the text-only representation of data. The system proposed in this chapter aims to support the developer during debugging by visualizing data in a graphic way and thus eliminating the cognitive effort needed to parse and interpret text based logging messages. The cognitive effort required during debugging can be further reduced by a flexible system which can be adapted to different preferences of different developers. Every developer can choose a visualization that fits their mental model of the debugged data best.


%%%% GOALS %%%%
\section{Goals}
The main goal of this work is to design a debugging system which is suitable for debugging in robotics and can be used to evaluate the hypothesis stated above. The design of the system is independent of a specific robotic framework and could be implemented for any available framework.
While most of the currently available visualization tools in robotics focus on pre-defined and spatial data to help understand the robot and the environment in which it runs \cite{Collett2010, Quigley2009}, rendering of abstract data is still uncommon. The design of the debugging system should be flexible enough to allow visualization of arbitrary and abstract data. Although abstract data will be the main focus of the debugging system since tools to visualize pre-defined data are already widely available, the designed system is not restricted to abstract data.

The system provides a graphical user interface which can be used by developers to visualize all kinds of data from the robotic application. The developers can customize the visualization according to the current robot hardware, development stage and personal preferences. The system can be adapted to many different use cases and should reduce the cognitive effort during debugging by visualizing the data according to the mental model of the developer and meaning of the data.

%%%% REQUIREMENTS %%%%
\section{Requirements}
The requirements of the flexible visual debugging system are mostly dominated by the special requirements of robotics. Some requirements are derived from the hypothesis and focus more on development performance and speed. This section presents the elicited requirements for a flexible visual debugging system.

\subsection{Distributed Live Debugging}
When debugging robotic applications it is usually not possible to interrupt the execution of the application and follow a step-through approach. All data must be captured, processed and visualized live. Robotic applications are usually distributed systems, because they often run on mobile robots or are so complex that multiple computers are used to make the system faster. This means the system must be able to handle communication distributed across a network.

\subsection{Adaptable Tool}
Due to many different application scenarios in robotics and the diverse environment of available frameworks for robot development, many researchers and developers have built their own tools to support them during debugging \cite{Collett2010}. Developing your own debugging tool is extremely time consuming and the developed tools are often one-time-only tools, because they don't fit the use case of other applications and are too hard to adapt to a new project.
In order to allow the use of the proposed debugging system in many different scenarios and use cases, flexibility has a high priority. Flexibility not only means the system can be adapted easily to fit different problems, it also means the system can be adapted to suit different developer's preferences. Each developer might prefer a different kind of visualization of the collected data, thus loose coupling of the collected data and the visualization widgets is necessary.

\subsection{Low Configuration Overhead}
Iterative development, small changes are deployed during development, setting up must be fast and easy and should not distract from the problem analysis task.

Debugging robotic applications is a highly iterative process, where small changes are made and immediately deployed to the robot. The debugging system should keep the configuration overhead as minimal as possible so that the developer is not distracted from the problem analysis task. Adding and removing new visualizations should not require many steps, otherwise the developer will be distracted by the configuration task.

[I'm talking about adding and removing visualizations but I haven't talked about the whole dashboard principle yet. Maybe I should add that to the goals?]

%%%% SYSTEM DESIGN %%%%
\section{System Design}
Discuss architecture and how it interacts with an eventual robot framework. All of them probably have some sort of inter process communication or a communication middleware in general, which can be accessed by the visualization tool.

\subsection{Computation Model / Architecture}
How ROS was used to achieve decoupled monitoring of data (topics)?

--> this would need to be separated from ros, but probably a general strategy for decoupling the collection and visualization of data can be found (worst case: listener thread?)

\subsection{Object Model}
As in the ICRA paper. Maybe some classes should be renamed to be more ROS unspecific (like Subscription).

\subsection{API}
ROSDashboard API, think about where to talk alternatives: extend ros logging framework, use tracepoints, etc.

--> too implementation specific, this is all ROS code. I don't know if this could be done in other frameworks as well (since I don't know them really), but in theory it should be possible to create a set of helper methods for every framework that uses the built in communication layer to have one-liner publish statements.

\subsection{Graphical User Interface}
The user interface for the debugging tool should allow to add and remove visualizations easily and arrange them however the developer wants. Thus a central dashboard approach was chosen, where visualizations can be positioned freely in form of widgets. Adding new visualizations to the dashboard can be done through a "Drag \& Drop" mechanism, once the widget is on the dashboard it can be resized and repositioned on the canvas. The initial mockup of the proposed graphical user interface is shown in figure~\ref{gui_mockup}.

\begin{figure}[htbp]
  \centering
  \includegraphics[width=\textwidth]{img/initial_gui_mockup.jpg}
  \caption{Paper mockup for the graphical interface.}
  \label{gui_mockup}
\end{figure}

Each widget represents one type of visiualization, this can be as simple as a dial for numberic values or more complex visualizations for more concrete data like a map visualization. Although the design of the user interface does not restrict the types of visualizations, it is intended for simple data, since other tools like RViz (see~\ref{rviz}) are more specialized for those use cases.

\begin{figure}[htbp]
  \centering
  \includegraphics[width=.5\textwidth]{img/initial_widget_mockup.jpg}
  \caption{Paper mockup for a visualization widget.}
  \label{widget_mockup}
\end{figure}

Figure~\ref{widget_mockup} shows an early mockup of a widget on the dashboard. This exemplary visualization widget has the form of a progress bar where numeric values can be visualized. The widgets can be hooked up to data providers and will be updated once a new value is available.

[Introduce data providers in the architecture. In ROS: subscriptions]

\chapter{ROSDashboard}

[should the chapter be renamed to "prototype" or "prototypical implementation"?]

Write about rosdashboard (overview).

rosdashboard is a prototypical implementation of the system presented in \ref{visual_debugging_system} for the ROS middleware / framework. ROS was chosen as middleware because the simple publish/subscribe mechanism allowed to connect to topics transparently without the other party explicitly knowing about the visualization tool.

[Not sure if this is possible in other frameworks, if not it could be stated that more work is needed for those frameworks in order to make communication transparent. In general I can argue that ROS fits the requierements best, the system design was not intended to fit every framework but was developed without compromises current systems might have]

[I'm not sure how to write this chapter: can I have a section "why ROS?", "how the requirements are met?", ...?]

\section{Implementation Details}

\subsection{Object Model}
As in the ICRA paper. Should this be part of system design and not implementation detail? Maybe skip the impl detail section and put everything in system design?

\subsection{Topic Introspection}

\begin{figure}[thpb]
  \centering
  \framebox{
    \includegraphics[scale=0.8]{img/topic_setup.png}
  }  
  \caption{Screenshot of the topic setup dialog. [re-do screenshot with rosdashboard in background]}
  \label{topic setup screenshot}
\end{figure}

ROS topics were originally not designed and developed as something the user or developer chooses graphically: They are usually created, configured and used in the source code. ROSDashboard exposes the topic setup in a graphical user interface every time a new widget is added to the dashboard. To make this as easy as possible and without much overhead, a technical solution was chosen to reduce the number of fields to be set during the topic subscription setup. Normally you have to select a topic name and a message data type. The data type can be one of the standard message types like Float, Integer, String and Boolean or a more complex message type which contains more information in a structured message. To access one data element of a message the ``datafield'' field was introduced in the graphical interface. Fig.~\ref{topic setup screenshot} shows an exemplary topic setup configuration to access the linear velocity of the \emph{/turtlesim/Velocity} message published to the topic \emph{/turtle1/command\_velocity}. Using Python's duck typing and the \emph{rostopic} module it was possible to avoid the complexity of dynamically binding message type classes during runtime and detect the message type automatically. If a topic is not yet published and thus the message type of this topic is not defined yet, the method call to \emph{rostopic} will block until the message type becomes available. To avoid blocking of the user interface a listener thread was implemented to wait until the message type for a topic becomes available (see Fig.~\ref{topic subscription}). Avoiding to manually ask the user for a message type makes the configuration of widgets easier and faster for the user, it also keeps the implementation significantly simpler, because no dynamic binding of message type classes during runtime is needed.

\begin{figure}[thpb]
  \centering
  \framebox{
    \includegraphics[scale=0.4]{diagrams/topic_subscription.pdf}
  }  
  \caption{Exemplary flow of events for asynchronous topic subscription setup.}
  \label{topic subscription}
\end{figure}

\subsection{Plugin Framework?}
I would need to implement that first...
\subsection{Example Plugin?}
Which methods need to be overwritten and how are callbacks implemented.

\chapter{Case Study}
Describe the case study: two different steps, the first one without code modification (monitoring of existing communication channels), the second one where code is actually modified to emit special debugging messages (can be done with rosdashboards api or manually).

Talk to Jamie Diprose about his Nao work with ROS (probably better than the turtlesim example from the ICRA paper), or see if rosbag data can be found and build showcase on top of the rosbag data.

\subsection{Case Study Configuration}
What machines, software versions, robots, etc were used.

\subsection{Without Code Modification}
\subsection{With Code Modification}

\subsection{Discussion}
Discuss the results from the case study?


\chapter{Future Work}
\label{future_work}

\section{Evaluation}

\chapter{Conclusion}


\appendix
\stepcounter{chapter}
\bibliographystyle{unsrt}
\bibliography{bibtex/bachelor_thesis,bibtex/bachelor_thesis_2}

\addcontentsline{toc}{chapter}{\protect\numberline{A}{Bibliography}}

%~\pagebreak
%~\stepcounter{chapter}
%~\listoffigures
%~\addcontentsline{toc}{chapter}{\protect\numberline{B}{List of
%~figures}}

%~\pagebreak
%~\stepcounter{chapter}
%~\listoftables
%~\addcontentsline{toc}{chapter}{\protect\numberline{C}{List of
%~tables}}

\end{document}
